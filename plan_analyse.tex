% Options for packages loaded elsewhere
\PassOptionsToPackage{unicode}{hyperref}
\PassOptionsToPackage{hyphens}{url}
\PassOptionsToPackage{dvipsnames,svgnames,x11names}{xcolor}
%
\documentclass[
  a4paper,
  french]{scrreprt}

\usepackage{amsmath,amssymb}
\usepackage{iftex}
\ifPDFTeX
  \usepackage[T1]{fontenc}
  \usepackage[utf8]{inputenc}
  \usepackage{textcomp} % provide euro and other symbols
\else % if luatex or xetex
  \usepackage{unicode-math}
  \defaultfontfeatures{Scale=MatchLowercase}
  \defaultfontfeatures[\rmfamily]{Ligatures=TeX,Scale=1}
\fi
\usepackage{lmodern}
\ifPDFTeX\else  
    % xetex/luatex font selection
\fi
% Use upquote if available, for straight quotes in verbatim environments
\IfFileExists{upquote.sty}{\usepackage{upquote}}{}
\IfFileExists{microtype.sty}{% use microtype if available
  \usepackage[]{microtype}
  \UseMicrotypeSet[protrusion]{basicmath} % disable protrusion for tt fonts
}{}
\makeatletter
\@ifundefined{KOMAClassName}{% if non-KOMA class
  \IfFileExists{parskip.sty}{%
    \usepackage{parskip}
  }{% else
    \setlength{\parindent}{0pt}
    \setlength{\parskip}{6pt plus 2pt minus 1pt}}
}{% if KOMA class
  \KOMAoptions{parskip=half}}
\makeatother
\usepackage{xcolor}
\setlength{\emergencystretch}{3em} % prevent overfull lines
\setcounter{secnumdepth}{5}
% Make \paragraph and \subparagraph free-standing
\ifx\paragraph\undefined\else
  \let\oldparagraph\paragraph
  \renewcommand{\paragraph}[1]{\oldparagraph{#1}\mbox{}}
\fi
\ifx\subparagraph\undefined\else
  \let\oldsubparagraph\subparagraph
  \renewcommand{\subparagraph}[1]{\oldsubparagraph{#1}\mbox{}}
\fi

\providecommand{\tightlist}{%
  \setlength{\itemsep}{0pt}\setlength{\parskip}{0pt}}\usepackage{longtable,booktabs,array}
\usepackage{calc} % for calculating minipage widths
% Correct order of tables after \paragraph or \subparagraph
\usepackage{etoolbox}
\makeatletter
\patchcmd\longtable{\par}{\if@noskipsec\mbox{}\fi\par}{}{}
\makeatother
% Allow footnotes in longtable head/foot
\IfFileExists{footnotehyper.sty}{\usepackage{footnotehyper}}{\usepackage{footnote}}
\makesavenoteenv{longtable}
\usepackage{graphicx}
\makeatletter
\def\maxwidth{\ifdim\Gin@nat@width>\linewidth\linewidth\else\Gin@nat@width\fi}
\def\maxheight{\ifdim\Gin@nat@height>\textheight\textheight\else\Gin@nat@height\fi}
\makeatother
% Scale images if necessary, so that they will not overflow the page
% margins by default, and it is still possible to overwrite the defaults
% using explicit options in \includegraphics[width, height, ...]{}
\setkeys{Gin}{width=\maxwidth,height=\maxheight,keepaspectratio}
% Set default figure placement to htbp
\makeatletter
\def\fps@figure{htbp}
\makeatother

\usepackage{booktabs}
\usepackage{longtable}
\usepackage{array}
\usepackage{multirow}
\usepackage{wrapfig}
\usepackage{float}
\usepackage{colortbl}
\usepackage{pdflscape}
\usepackage{tabu}
\usepackage{threeparttable}
\usepackage{threeparttablex}
\usepackage[normalem]{ulem}
\usepackage{makecell}
\usepackage{xcolor}
    \definecolor{novo}{HTML}{27484b}
    \usepackage[locale = FR, 
                per-mode = symbol, 
                range-phrase = { à },
                range-units = single]{siunitx}
    \usepackage[section]{placeins}
    \usepackage{arsenal}
    \usepackage{hyperref}
    \usepackage{tcolorbox}
    \usepackage[nonumberlist, 
                automake,
                style=treegroup]{glossaries-extra}
    \usepackage{glossaries-babel}
    \setabbreviationstyle{long-short-sc}
    \newglossary*{stat}{Statistiques}
    \loadglsentries{listacro}
    \makeglossaries
    \newcommand{\doc}{D\textsuperscript{r} }
    \newcommand{\mme}{M\textsuperscript{me} }
    %
    \tcbset{colback=novo!5!white,colframe=novo!75!black}

\newtcbox{\boxphm}{colback=novo!5!white, 
                       width = 5cm,
                       %text width=0.9\linewidth,
                       halign = justify}
\makeatletter
\@ifpackageloaded{caption}{}{\usepackage{caption}}
\AtBeginDocument{%
\ifdefined\contentsname
  \renewcommand*\contentsname{Table des matières}
\else
  \newcommand\contentsname{Table des matières}
\fi
\ifdefined\listfigurename
  \renewcommand*\listfigurename{Liste des Figures}
\else
  \newcommand\listfigurename{Liste des Figures}
\fi
\ifdefined\listtablename
  \renewcommand*\listtablename{Liste des Tables}
\else
  \newcommand\listtablename{Liste des Tables}
\fi
\ifdefined\figurename
  \renewcommand*\figurename{Figure}
\else
  \newcommand\figurename{Figure}
\fi
\ifdefined\tablename
  \renewcommand*\tablename{Table}
\else
  \newcommand\tablename{Table}
\fi
}
\@ifpackageloaded{float}{}{\usepackage{float}}
\floatstyle{ruled}
\@ifundefined{c@chapter}{\newfloat{codelisting}{h}{lop}}{\newfloat{codelisting}{h}{lop}[chapter]}
\floatname{codelisting}{Listing}
\newcommand*\listoflistings{\listof{codelisting}{Liste des Listings}}
\makeatother
\makeatletter
\makeatother
\makeatletter
\@ifpackageloaded{caption}{}{\usepackage{caption}}
\@ifpackageloaded{subcaption}{}{\usepackage{subcaption}}
\makeatother

\usepackage{hyphenat}
\usepackage{ifthen}
\usepackage{calc}
\usepackage{calculator}

\usepackage{graphicx}
\usepackage{wallpaper}

\usepackage{geometry}

\usepackage{graphicx}
\usepackage{geometry}
\usepackage{afterpage}
\usepackage{tikz}
\usetikzlibrary{calc}
\usetikzlibrary{fadings}
\usepackage[pagecolor=none]{pagecolor}


% Set the titlepage font families







% Set the coverpage font families

\ifLuaTeX
\usepackage[bidi=basic]{babel}
\else
\usepackage[bidi=default]{babel}
\fi
\babelprovide[main,import]{french}
% get rid of language-specific shorthands (see #6817):
\let\LanguageShortHands\languageshorthands
\def\languageshorthands#1{}
\ifLuaTeX
  \usepackage{selnolig}  % disable illegal ligatures
\fi
\usepackage[]{biblatex}
\addbibresource{stat.bib}
\usepackage{bookmark}

\IfFileExists{xurl.sty}{\usepackage{xurl}}{} % add URL line breaks if available
\urlstyle{same} % disable monospaced font for URLs
\hypersetup{
  pdftitle={TRAP},
  pdfauthor={Dr Philippe MICHEL},
  pdflang={fr},
  colorlinks=true,
  linkcolor={blue},
  filecolor={Maroon},
  citecolor={Blue},
  urlcolor={Blue},
  pdfcreator={LaTeX via pandoc}}

\title{TRAP}
\usepackage{etoolbox}
\makeatletter
\providecommand{\subtitle}[1]{% add subtitle to \maketitle
  \apptocmd{\@title}{\par {\large #1 \par}}{}{}
}
\makeatother
\subtitle{Plan d'analyse statistique}
\author{Dr Philippe MICHEL}
\date{}

\begin{document}
%%%%% begin titlepage extension code


\begin{titlepage}

%%% TITLE PAGE START

% Set up alignment commands
%Page
\newcommand{\titlepagepagealign}{
\ifthenelse{\equal{left}{right}}{\raggedleft}{}
\ifthenelse{\equal{left}{center}}{\centering}{}
\ifthenelse{\equal{left}{left}}{\raggedright}{}
}


\newcommand{\titleandsubtitle}{
% Title and subtitle
{\textcolor{novo}{\Huge{\bfseries{\nohyphens{TRAP}}}}\par
}%

\vspace{\betweentitlesubtitle}
{
\textcolor{novo}{\huge{\nohyphens{Plan d'analyse statistique}}}\par
}}
\newcommand{\titlepagetitleblock}{
\titleandsubtitle
}

\newcommand{\authorstyle}[1]{{\large{#1}}}

\newcommand{\affiliationstyle}[1]{{\large{#1}}}

\newcommand{\titlepageauthorblock}{
{\authorstyle{\nohyphens{Dr Philippe MICHEL}{\textsuperscript{1}}}}}

\newcommand{\titlepageaffiliationblock}{
\hangindent=1em
\hangafter=1
{\affiliationstyle{
{1}.~Hôpital NOVO,~Unité de Soutien à la Recherche Clinique


\vspace{1\baselineskip} 
}}
}
\newcommand{\headerstyled}{%
{}
}
\newcommand{\footerstyled}{%
{\large{\today}}
}
\newcommand{\datestyled}{%
{}
}


\newcommand{\titlepageheaderblock}{\headerstyled}

\newcommand{\titlepagefooterblock}{
\footerstyled
}

\newcommand{\titlepagedateblock}{
\datestyled
}

%set up blocks so user can specify order
\newcommand{\titleblock}{\newlength{\betweentitlesubtitle}
\setlength{\betweentitlesubtitle}{\baselineskip}
{

{\titlepagetitleblock}
}

\vspace{4\baselineskip}
}

\newcommand{\authorblock}{{\titlepageauthorblock}

\vspace{2\baselineskip}
}

\newcommand{\affiliationblock}{{\titlepageaffiliationblock}

\vspace{1pt}
}

\newcommand{\logoblock}{}

\newcommand{\footerblock}{{\titlepagefooterblock}

\vspace{1pt}
}

\newcommand{\dateblock}{}

\newcommand{\headerblock}{}
\newgeometry{top=3in,bottom=1in,right=1in,left=1in}
% background image
\newlength{\bgimagesize}
\setlength{\bgimagesize}{0.5\paperwidth}
\LENGTHDIVIDE{\bgimagesize}{\paperwidth}{\theRatio} % from calculator pkg
\ThisULCornerWallPaper{\theRatio}{novo\_usrc.png}

\thispagestyle{empty} % no page numbers on titlepages


\newcommand{\vrulecode}{\rule{\vrulewidth}{\textheight}}
\newlength{\vrulewidth}
\setlength{\vrulewidth}{0.1cm}
\newlength{\B}
\setlength{\B}{\ifdim\vrulewidth > 0pt 0.05\textwidth\else 0pt\fi}
\newlength{\minipagewidth}
\ifthenelse{\equal{left}{left} \OR \equal{left}{right} }
{% True case
\setlength{\minipagewidth}{\textwidth - \vrulewidth - \B - 0.1\textwidth}
}{
\setlength{\minipagewidth}{\textwidth - 2\vrulewidth - 2\B - 0.1\textwidth}
}
\ifthenelse{\equal{left}{left} \OR \equal{left}{leftright}}
{% True case
\raggedleft % needed for the minipage to work
\vrulecode
\hspace{\B}
}{%
\raggedright % else it is right only and width is not 0
}
% [position of box][box height][inner position]{width}
% [s] means stretch out vertically; assuming there is a vfill
\begin{minipage}[b][\textheight][s]{\minipagewidth}
\titlepagepagealign
\titleblock

\authorblock

\affiliationblock

\vfill

\logoblock

\footerblock
\par

\end{minipage}\ifthenelse{\equal{left}{right} \OR \equal{left}{leftright} }{
\hspace{\B}
\vrulecode}{}
\clearpage
\restoregeometry
%%% TITLE PAGE END

\input{generique.tex}
\clearpage
\end{titlepage}
\setcounter{page}{1}

%%%%% end titlepage extension code

\renewcommand*\contentsname{Table des matières}
{
\hypersetup{linkcolor=}
\setcounter{tocdepth}{2}
\tableofcontents
}
\chapter{Généralités}\label{guxe9nuxe9ralituxe9s}

Le \gls{alpha} retenu sera de \num{0.05} \& la \gls{puissance} de
\num{0.8}.

Les variables numériques seront présentées par leur moyenne avec
l'écart-type \& comparées grâce au test de Student. Les variables
discrètes seront présentés en nombre avec le pourcentage. L'intervalle
de confiance (à \qty{95}{\percent}) sera calculé par \gls{bootstrap}
(package \texttt{boot} \autocite{boot}). Le test du \(\chi^2\) de
Spearman sera utilisé sous réserve d'effectifs suffisants, à défaut le
test exact de Fischer. Des graphiques seront réalisés pour les résultats
importants (package \texttt{ggplot2} \autocite{ggplot}).

\section{Taille de l'échantillon}\label{taille-de-luxe9chantillon}

Il s'agit d'une étude observationnelle simple pour lesquelles il est
difficile de préciser un nombre de cas nécessaires en l'absence de test
statistique. Néanmoins on peut estimer à au moins 110 cas le strict
minimum nécessaire pour avoir un échantillon utilisable s'il n'y a pas
trop de données manquantes (\gls{alpha} \num{0.05}, marge d'erreur
admissible \qty{10}{\percent}).

\section{Données manquantes}\label{donnuxe9es-manquantes}

Le décompte des données manquantes sera réalisé \& présenté par un
tableau ou un graphique. Les variables comportant trop de données
manquantes ou non utilisables ne seront pas prises en compte après
validation par le promoteur.

Après ce premier tri une imputation des données manquantes (package
\texttt{missMDA} \autocite{missmda}) sera réalisée uniquement pour
l'analyse factorielle \& pour la recherche du meilleur modèle par
méthode de \gls{lasso} pour les analyses par régression (logistique ou
linéaire). Néanmoins pour cette analyse, après choix du modèle, le
calcul final sera fait avec les données réelles.

\section{Qualité des données}\label{qualituxe9-des-donnuxe9es}

Une analyse de corrélation (package \texttt{corr} \autocite{corr}) entre
les variables sera réalisée \& présentée sur un graphique de
corrélation. Si certaines variables se montrent anormalement corrélées
elles devront être exclues de l'étude après accord du promoteur.

Les variables monotones (toutes les valeurs identiques) ne seront pas
analysées mais seront signalées.

Les variables en texte libre ne seront pas analysées.

\chapter{Description de la
population}\label{description-de-la-population}

\section{Analyse simple}\label{analyse-simple}

Un tableau présentera les données démographiques des patients
(âge,sexe\dots) \& les circonstances du traumatisme.

\section{Analyse factorielle}\label{analyse-factorielle}

Si le nombre de cas recueillis le permet une analyse factorielle en MCA
(Analyse de correspondances multiples -- package \texttt{FactoMineR}
\autocite{facto}) sera réalisée.

Cette analyse ne pourra être réalisée qu'après transformation des
variables numériques en catégories \& imputation des données manquantes
ce qui n'est possible que si ces dernières ne sont pas trop nombreuses.

\chapter{Objectif principal}\label{objectif-principal}

\begin{tcolorbox}
Présence, lors de la prise en charge pré-hospitalière d’un critère de transfusion sanguine massive et immédiate mesurée par le score ABC entre l’arrivée du \gls{smur} et \qty{10}{\min} avant l’arrivée à l’hôpital.
\end{tcolorbox}

Le score ABC sera calculé pour tous les patients. Un premier tableau
présentera la répartition des scores ABC.

Le nombre avec le pourcentage (et son intervalle de confiance à 95 \%)
de patients ayant un score ABC \(\geq\) 2 sera présenté pour tous les
stades (temps) de la prise en charge.

\chapter{Objectifs secondaires}\label{objectifs-secondaires}

\section{Objectif secondaire 1}\label{objectif-secondaire-1}

\begin{tcolorbox}
La présence d’au moins une transfusion, quel que soit le volume entre le bilan et l’arrivée à l’hôpital.
\end{tcolorbox}

Le nombre avec le pourcentage (avec son intervalle de confiance à
\qty{95}{\percent}) de patients ayant reçu une transfusion sera
présenté, croisé avec le stade de la prise en charge (temps) \& le score
ABC.

\section{Objectif secondaire 2}\label{objectif-secondaire-2}

\begin{tcolorbox}
La mortalité est définie par les patients vivant à l’arrivée du \gls{smur} et décédés à l’arrivée à l’hôpital.
\end{tcolorbox}

Le nombre avec le pourcentage (avec son intervalle de confiance à
\qty{95}{\percent}) de patients décédés sera présenté, croisé avec le
score ABC.

\section{Objectif secondaire 3}\label{objectif-secondaire-3}

\begin{tcolorbox}
Le temps de prise en charge est déterminé par le délai entre l’arrivée de l’équipe médicale et l’arrivée à l’hôpital.
\end{tcolorbox}

La durée de la prise en charge sera présentée en minutes (moyenne,
écart-types, extrêmes).

\section{Objectif secondaire 4}\label{objectif-secondaire-4}

\begin{tcolorbox}
Nombre de patients potentiellement éligibles à la pose d’un \gls{reboa} au regard de la présence de saignement abdominal ou pelvien à l’échographie, d’une \gls{pas} > \qty{90}{\mmHg} malgré \qty{3}{\mg\per\hour} de \gls{nad}, l’absence de trauma thoracique ou des \gls{tsa}.
\end{tcolorbox}

Le nombre avec le pourcentage (avec son intervalle de confiance à
\qty{95}{\percent}) de patients éligibles à la pose d'un \gls{reboa}
sera présenté.

\chapter{Technique}\label{technique}

L'analyse statistique sera réalisée avec le logiciel
\textbf{R}\autocite{rstat} \& divers packages. Outre ceux cités dans le
texte ou utilisera en particulier \texttt{tidyverse} \autocite{tidy} \&
\texttt{baseph} \autocite{baseph}.

Un dépôt GitHub sera utilisé qui ne comprendra que le code \& non les
données ou résultats. Au besoin un faux tableau de données sera présenté
pour permettre des tests.

\url{https://github.com/philippemichel/TRAP}

\printglossary[ title={Abréviations \& Acronymes}, toctitle={Abréviations \& Acronymes}]

\printglossary[type = stat, title={Glossaire statistique}, toctitle={Glossaire statistique}]

\addcontentsline{toc}{chapter}{Bibliographie}


\printbibliography


\end{document}
